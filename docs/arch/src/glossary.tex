\newacronym{uxd}{UXD}{User eXperience Design}
\newacronym{ux}{UX}{User eXperience}
\newacronym{ui}{UI}{User Interface}

\newacronymentry{ai}{AI}{Artificial Intelligence}{%
A branch of Computer Science that studies the design
of intelligent systems.
\wref{Artificial_Intelligence}}{}

\newacronymentry{mvc}{MVC}{Model-View-Controller}{%
A software architecture design pattern that is applied
to build a reactive system. MVC is often used to
loosely couple presentational aspects of a software
application to pure abstractions that represent
the application semantics.
\wref{Model-view-controller}
}{user1={the MVC design pattern}}

\newacronymentry{adt}{ADT}{Abstract Data Type}{%
A means of abstracting a concept with well-defined semantics. An ADT acts
as a bridge between the semantics of an abstracted concept and the physical 
representation thereof in a computer.
\wref{Abstract_data_type}
}{}

\newacronymentry{io}{I/O}{Input/Output}{%
The communication between two \glspl{agent}. \emph{I/O} 
is a general term to refer to any method that two 
\glspl{agent} use to either receive input from
 or pass data to each other.
\wref{Input/output}
}{user1={the I/O mechanism}, user2={an I/O mechanism}}

\newacronymentry{xml}{XML}{Extensible Markup Language}{%
An open and widely supported standard that 
proposes a syntax in which to encode data with
arbitrary semantics.
\wref{XML}
}{}

\newacronymentry{xslt}{XSLT}{Extensible Stylesheet Language Transformation}{%
An \gls{xml}-based domain-specific language that is used for transformation
of \gls{xml} documents into other text-based formats.
\wref{XSLT}
}{user6={XSLT stylesheet}, see={xml,xpath}}

\newacronymentry{xpath}{XPath}{XML Path Language}{%
A domain-specific query language for selecting nodes from an \gls{xml}.
In addition, XPath computes values --e.g. strings, numbers-- from the
content of \gls{xml} documents.
\wref{XPath}
}{see={xml,xslt}}

\newacronymentry{jpeg}{JPEG}{Joint Photographic Experts Group}{%
An open standard for coding and compressing still digital images, 
named after the committee that overseas the development of the standard 
\cite{international_telecommunication_union_information_1993}.
\wref{JPEG}
}{}

\newacronymentry{lisp}{Lisp}{List Programming}{%
A class of programming languages in which code is treated as data.
In this document, we are primarily interested in a particular
dialect of Lisp, namely \gls{racket}.
\wref{Lisp_(programming_language)}
}{first={Lisp}, see={scheme,racket}}

\newacronymentry{gui}{GUI}{Graphical User Interface}{%
\Glsuserii{io} that uses digital graphics to output information.
GUIs receive input by having users directly interact with the
graphics they display.
\wref{Graphical_user_interface}
}{%
plural={GUIs}, 
user1={the GUI},
user3={the graphical user interface},
see={io}}

\newacronymentry{repl}{REPL}{Read-Eval-Print Loop}{%
\Glsuserii{io} that receives the lexical representation of a program (code),
evaluates the code based on well-defined semantics, and prints out the 
result of the evaluation of the code. Unless the code is well-formed and sound,
the result will be an error.
\wref{Read-eval-print_loop}
}{user1={the REPL},see={io}}

\newacronymentry{vm}{VM}{Virtual Machine}{%
A computer program that abstracts the underlying hardware. 
Examples include compilers and interpreters}{}

\newglossaryentry{jpegbd}{name={JPEG Baseline Decoder},
user2={a JPEG baseline decoder}, description={%
\gls{jpeg} Given a sequence of bytes that conforms to the 
}}

\newglossaryentry{agent}{name={Agent}, plural={agents},
user1={the agent}, user2={an agent}, description={%
Any entity such as a human or a computer program that is capable of 
communicating with \gls{neille}.
\wref{Software_agent}
}}

\newglossaryentry{scheme}{name={Scheme}, 
user1={the Scheme programming language}, description={%
A dialect of the Lisp family of programming languages. Among 
other things, Scheme is dynamically-typed, statically-scoped, and has 
first-class functions and continuations. Scheme is the 
predecessor of \gls{racket}.
\wref{Scheme_(programming_language)}
},see={lisp,racket}}

\newglossaryentry{racket}{name={Racket}, 
user1={the Racket programming language},
user2={the Racket interpreter},
user3={the Racket platform}, 
user4={the Racket virtual machine},
user5={the Racket dialect of Lisp},
description={%
A multi-paradigm programming language that also serves as a
platform for language creation, design, and implementation.
The programming language itself is inspired by \gls{scheme}.
However, the underlying platform enables language designers
to create languages with arbitrary syntax and semantics.
\wref{Racket_(programming_language)}
},see={lisp,scheme}}

\newglossaryentry{facebook}{name={Facebook},
user1={the Facebook web-based computing platform},
description={A propriatary, and distributed platform for 
social interactions on the Internet.
\wref{Facebook}
}}

\newglossaryentry{ws}{name={Warstorm},
user1={the Warstorm card game}, description={%
An online collectible card game that is primarily distributed over
\gls{facebook}.
\newline
\newline
For more information on the Warstorm card game, please refer to:
\newline
\url{http://warstorm.wikia.com/wiki/Warstorm_Wiki}
}}

\newglossaryentry{neille}{name={Neille}, 
user1={the Neille project}, description={%
A \gls{ws}-like card game written in \glsuserv{racket}
},see={ws,racket}}

\newglossaryentry{reify}{name={Reification}, text={reify}, description={%
Turning an abstract grammar entity to a concrete grammar entity.\newline
Antonym: \gls{deify}}}

\newglossaryentry{deify}{name={Deification}, text={deify}, description={%
Transforming an instance of the concrete grammar into an instance of the
abstract grammar.\newline
Synonym: \emph{Parsing}. Antonym: \gls{reify}}}

\newglossaryentry{nxml}{name={nXML mode}, first={nXML mode for Emacs},
description={%
An addon for GNU Emacs that turns GNU Emacs into an exceptionally
powerful \gls{xml} editor.
\newline
\newline
For more information on nXML mode, please refer to:
\newline
  \url{http://www.thaiopensource.com/nxml-mode/}
}}

\newglossaryentry{rng}{name={RELAX NG}, 
user1={the RELAX NG schema language}, description={%
.
\wref{RELAX_NG}
}}
